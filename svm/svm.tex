\documentclass{article}

\usepackage[top=0.5cm, bottom=0.5cm, left=0.5cm, right=0.5cm,landscape]{geometry}

\usepackage{url}
\usepackage{multicol}
\usepackage{amsfonts,amsmath,amssymb}

\usepackage{colortbl}
\usepackage{xcolor}
\usepackage{mathtools}

\usepackage[utf8]{inputenc}
\usepackage{parskip}

\DeclareMathOperator{\argmin}{argmin}
\DeclareMathOperator{\argmax}{argmax}
\DeclareUnicodeCharacter{2716}{✖}

\begin{document}

\begin{center}{\huge{\textbf{Support Vector Machines cheatsheet}}}\\
	{\large By Jérémy Fix, Hervé Frezza-Buet}
\end{center}
\begin{multicols*}{3}

\section*{Kernels}
The separator reads
\begin{align*}
	h_{w, b}(x) = \sum_i w_i k(x_i, x) + b
\end{align*}
\begin{center}
	\begin{tabular}{lp{5cm} c}
		\textit{Linear} & $k(x, x') = <x, x'>$ \\
		\hline
		\textit{RBF} & $k_\sigma(x, x') = \exp(-\frac{\|x - x'\|^2}{2\sigma^2})$ \\ 
		\hline
		\textit{Polynomial} & $k_{\gamma, c, d}(x, x') = (\gamma <x, x'> + c)^d$ \\ 
	\end{tabular}
\end{center}

\section*{Classification SVM : C-SVC}
Primal optimization problem:
\begin{flalign*}
	\underset{w, b \xi}{\argmin}& \frac{1}{2} \|w\|^2 + C \mathbf{1}^T \xi\\
	\mbox{subject to}& \left\{\begin{array}{ll} 
			y_i h_{w, b}(x_i) \geq 1 - \xi_i &,  \forall (x_i, y_i) \in S \\
			\xi_i \geq 0&, \forall i
	\end{array}\right.
		\end{flalign*} 
		Dual optimization problem:
		\begin{flalign*}
			\underset{\alpha}{\argmin}& \frac{1}{2} \alpha^T Q \alpha - \mathbf{1}^T \alpha\\
			\mbox{subject to}& \left\{\begin{array}{ll} 
					y^T \alpha = 0\\
					0 \leq \alpha_i \leq C, \forall i
			\end{array}\right.\\
			Q_{i,j} = y_i y_j k(x_i, x_j)
				\end{flalign*} 
				The optimal separator reads :
				\begin{align*}
					h(x) = \sum_i \alpha_i y_i k(x_i, x) + b
				\end{align*}
				For the optimal separator, we have either :
				\begin{itemize}
					\item $\alpha_i = 0$ for \textbf{non support vector} $x_i$, the associated sample does not contribute to the definition of the separating hyperplane,
					\item $0 < \alpha_i < C$ for \textbf{support vectors} $x_i$ just on the margin. These samples are associated with $\xi_i = 0$ and $y_i h_{w, b}(x_i) = 1$  
					\item $\alpha_i = C$ for \textbf{support vectors} $x_i$ which involve non null slack variables; they are "on the wrong side" of the margin. These samples are associated with $\xi_i > 0$ and $y_i h_{w, b}(x_i) = 1 - \xi_i$ 
				\end{itemize}	
				\vfill\null
				\columnbreak

				On the figure below, we see a bi-class problem solved with a C-SVC with C=1.
				The positive samples are drawn with a white dot; The negative samples with a gray dot.\\

				\includegraphics[width=\columnwidth]{csvc001.png}

The support vectors are the samples indicated with a cross. A red cross indicates the samples on the margin. A black cross indicates the samples with non null slack variables $\xi$. Only the support vectors (i.e. the samples indicated with a cross) contribute to the definition of the separator.
			
\section*{C-SVC examples}

\begin{minipage}{0.5\columnwidth}
	400 samples, noise=0.25\\
	$C=0.1 ,\sigma=0.4$\\
	With a small C, a lot of vectors can be support vectors
	\columnbreak
\end{minipage}
\begin{minipage}{0.4\columnwidth}
	\includegraphics[width=\columnwidth]{csvc003.png}
\end{minipage}

\begin{minipage}{0.5\columnwidth}
	100 samples, noise=0.1\\
	$C=1000 ,\sigma=0.4$\\
	It costs a lot to add slack variables. The separator is really stuck to the data.\columnbreak  
\end{minipage}
\begin{minipage}{0.4\columnwidth}
\includegraphics[width=\columnwidth]{csvc002.png}
\end{minipage}

\columnbreak
\section*{Classification SVM : Nu-SVC}

Primal optimization problem:
\begin{flalign*}
	\underset{w, b, \xi, \rho}{\argmin}& \frac{1}{2} \|w\|^2 - \nu \rho + \frac{1}{|S|}\mathbf{1}^T \xi\\
	\mbox{subject to}& \left\{\begin{array}{ll} 
			y_i h_{w, b}(x_i) \geq \rho - \xi_i &,  \forall (x_i, y_i) \in S \\
			\rho \geq 0 &\\
			\xi_i \geq 0&, \forall i
\end{array}\right.
\end{flalign*} 
Dual optimization problem:
\begin{flalign*}
	\underset{w, b, \xi, \rho}{\argmin}& \frac{1}{2} \alpha^T Q \alpha  \\
	\mbox{subject to}& \left\{\begin{array}{ll} 
			y^T \alpha = 0& \\
			0 \leq \alpha_i \leq \frac{1}{|S|},& \forall i\\
			\mathbf{1}^T \alpha = \nu & 
\end{array}\right.
\end{flalign*} 


\end{multicols*}
\end{document}
